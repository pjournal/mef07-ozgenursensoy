% Options for packages loaded elsewhere
\PassOptionsToPackage{unicode}{hyperref}
\PassOptionsToPackage{hyphens}{url}
\PassOptionsToPackage{dvipsnames,svgnames,x11names}{xcolor}
%
\documentclass[
  letterpaper,
  DIV=11,
  numbers=noendperiod]{scrreprt}

\usepackage{amsmath,amssymb}
\usepackage{iftex}
\ifPDFTeX
  \usepackage[T1]{fontenc}
  \usepackage[utf8]{inputenc}
  \usepackage{textcomp} % provide euro and other symbols
\else % if luatex or xetex
  \usepackage{unicode-math}
  \defaultfontfeatures{Scale=MatchLowercase}
  \defaultfontfeatures[\rmfamily]{Ligatures=TeX,Scale=1}
\fi
\usepackage{lmodern}
\ifPDFTeX\else  
    % xetex/luatex font selection
\fi
% Use upquote if available, for straight quotes in verbatim environments
\IfFileExists{upquote.sty}{\usepackage{upquote}}{}
\IfFileExists{microtype.sty}{% use microtype if available
  \usepackage[]{microtype}
  \UseMicrotypeSet[protrusion]{basicmath} % disable protrusion for tt fonts
}{}
\makeatletter
\@ifundefined{KOMAClassName}{% if non-KOMA class
  \IfFileExists{parskip.sty}{%
    \usepackage{parskip}
  }{% else
    \setlength{\parindent}{0pt}
    \setlength{\parskip}{6pt plus 2pt minus 1pt}}
}{% if KOMA class
  \KOMAoptions{parskip=half}}
\makeatother
\usepackage{xcolor}
\setlength{\emergencystretch}{3em} % prevent overfull lines
\setcounter{secnumdepth}{5}
% Make \paragraph and \subparagraph free-standing
\ifx\paragraph\undefined\else
  \let\oldparagraph\paragraph
  \renewcommand{\paragraph}[1]{\oldparagraph{#1}\mbox{}}
\fi
\ifx\subparagraph\undefined\else
  \let\oldsubparagraph\subparagraph
  \renewcommand{\subparagraph}[1]{\oldsubparagraph{#1}\mbox{}}
\fi


\providecommand{\tightlist}{%
  \setlength{\itemsep}{0pt}\setlength{\parskip}{0pt}}\usepackage{longtable,booktabs,array}
\usepackage{calc} % for calculating minipage widths
% Correct order of tables after \paragraph or \subparagraph
\usepackage{etoolbox}
\makeatletter
\patchcmd\longtable{\par}{\if@noskipsec\mbox{}\fi\par}{}{}
\makeatother
% Allow footnotes in longtable head/foot
\IfFileExists{footnotehyper.sty}{\usepackage{footnotehyper}}{\usepackage{footnote}}
\makesavenoteenv{longtable}
\usepackage{graphicx}
\makeatletter
\def\maxwidth{\ifdim\Gin@nat@width>\linewidth\linewidth\else\Gin@nat@width\fi}
\def\maxheight{\ifdim\Gin@nat@height>\textheight\textheight\else\Gin@nat@height\fi}
\makeatother
% Scale images if necessary, so that they will not overflow the page
% margins by default, and it is still possible to overwrite the defaults
% using explicit options in \includegraphics[width, height, ...]{}
\setkeys{Gin}{width=\maxwidth,height=\maxheight,keepaspectratio}
% Set default figure placement to htbp
\makeatletter
\def\fps@figure{htbp}
\makeatother

\KOMAoption{captions}{tableheading}
\makeatletter
\makeatother
\makeatletter
\@ifpackageloaded{bookmark}{}{\usepackage{bookmark}}
\makeatother
\makeatletter
\@ifpackageloaded{caption}{}{\usepackage{caption}}
\AtBeginDocument{%
\ifdefined\contentsname
  \renewcommand*\contentsname{Table of contents}
\else
  \newcommand\contentsname{Table of contents}
\fi
\ifdefined\listfigurename
  \renewcommand*\listfigurename{List of Figures}
\else
  \newcommand\listfigurename{List of Figures}
\fi
\ifdefined\listtablename
  \renewcommand*\listtablename{List of Tables}
\else
  \newcommand\listtablename{List of Tables}
\fi
\ifdefined\figurename
  \renewcommand*\figurename{Figure}
\else
  \newcommand\figurename{Figure}
\fi
\ifdefined\tablename
  \renewcommand*\tablename{Table}
\else
  \newcommand\tablename{Table}
\fi
}
\@ifpackageloaded{float}{}{\usepackage{float}}
\floatstyle{ruled}
\@ifundefined{c@chapter}{\newfloat{codelisting}{h}{lop}}{\newfloat{codelisting}{h}{lop}[chapter]}
\floatname{codelisting}{Listing}
\newcommand*\listoflistings{\listof{codelisting}{List of Listings}}
\makeatother
\makeatletter
\@ifpackageloaded{caption}{}{\usepackage{caption}}
\@ifpackageloaded{subcaption}{}{\usepackage{subcaption}}
\makeatother
\makeatletter
\@ifpackageloaded{tcolorbox}{}{\usepackage[skins,breakable]{tcolorbox}}
\makeatother
\makeatletter
\@ifundefined{shadecolor}{\definecolor{shadecolor}{rgb}{.97, .97, .97}}
\makeatother
\makeatletter
\makeatother
\makeatletter
\makeatother
\ifLuaTeX
  \usepackage{selnolig}  % disable illegal ligatures
\fi
\IfFileExists{bookmark.sty}{\usepackage{bookmark}}{\usepackage{hyperref}}
\IfFileExists{xurl.sty}{\usepackage{xurl}}{} % add URL line breaks if available
\urlstyle{same} % disable monospaced font for URLs
\hypersetup{
  pdftitle={Assignment 1},
  pdfauthor={Özgenur Şensoy},
  colorlinks=true,
  linkcolor={blue},
  filecolor={Maroon},
  citecolor={Blue},
  urlcolor={Blue},
  pdfcreator={LaTeX via pandoc}}

\title{Assignment 1}
\author{}
\date{2022-10-01}

\begin{document}
\maketitle
\ifdefined\Shaded\renewenvironment{Shaded}{\begin{tcolorbox}[interior hidden, frame hidden, sharp corners, breakable, enhanced, boxrule=0pt, borderline west={3pt}{0pt}{shadecolor}]}{\end{tcolorbox}}\fi

\renewcommand*\contentsname{Table of contents}
{
\hypersetup{linkcolor=}
\setcounter{tocdepth}{2}
\tableofcontents
}
\bookmarksetup{startatroot}

\hypertarget{introduction}{%
\chapter*{Introduction}\label{introduction}}
\addcontentsline{toc}{chapter}{Introduction}

\markboth{Introduction}{Introduction}

This progress journal covers {[}Özgenur Şensoy / PROJECT GROUP NAME
haven't decided yet{]}'s work during their term at
\href{https://mef-bda503.github.io/fall22/}{BDA 503 Fall 2022}.

Each section is an assignment or an individual work.

\bookmarksetup{startatroot}

\hypertarget{assignment-1}{%
\chapter{Assignment 1}\label{assignment-1}}

\bookmarksetup{startatroot}

\hypertarget{about-me}{%
\chapter{\textbar{} About Me}\label{about-me}}

{Hello folks!}

I'm Özgenur Şensoy, MSc in Big Data Analytics. I graduated from MEF
University from the Department of Elementary Mathematics Education.
Since I did not feel like I belonged there during my internships, I
switched to data analytics where I could be more productive. I want to
put my data science skills into practice in a few companies that I dream
of. In addition, I want to progress by nourishing my knowledge with
courses on artificial intelligence.

\begin{figure}

{\centering 

\href{https://www.linkedin.com/in/özgenurşensoy/}{\includegraphics{index_files/mediabag/linkedin--0077B5.svg--style=for-the-badge-logo=linkedin-}}

}

\caption{LinkedIn Badge}

\end{figure}

\bookmarksetup{startatroot}

\hypertarget{yotube-video}{%
\chapter{\textbar{} Yotube Video}\label{yotube-video}}

\hypertarget{data-visualization}{%
\subsection{Data Visualization}\label{data-visualization}}

\href{https://www.youtube.com/watch?v=5zJC0AB-UK8\&list=PL9HYL-VRX0oTOK4cpbCbRk15K2roEgzVW\&index=21}{\emph{Teaching
R online with RStudio Cloud}}

I chose to watch the video titled \textbf{``Data visualization and
plotting with Shiny for Python \textbar\textbar{} Carson Sievert
\textbar\textbar{} RStudio''} The video covers how to create interactive
data visualization using Shiny, Python, and Pandas. Here are the key
topics:

\textbf{Layout and Slider:} At the beginning of the video, we learn how
to lay out a user interface and add interactive elements such as sliders
using Shiny. This allows users to interact with data or graphs.

\textbf{Reactive Calculations :}The term ``Reactive calculations''
refers to the ability of users to make calculations based on data. For
example, you can create a structure where the calculations are
automatically updated when a user changes the slider.

\textbf{Render Plot:} Specifies how to render and update graphics or
plots to be displayed to the user. Data-driven charts can refresh
automatically as users interact.

\textbf{Pandas Plotting:} Pandas is a widely used library for data
analysis and manipulation in Python. The video could possibly also cover
how to analyze data using DataFrames and how to use Pandas' plotting
capabilities.

The video explains how you can create an app with an interactive
graphic. This implementation seems to change the number of bars in the
calculated histogram when you change a slider. The video shows you how
to layout such interactive applications using ``layout sidebar'', a
subpackage of Shiny. This allows users to make the data and chart more
interactive. Additionally, ``reactive calculations'' and ``render plot''
topics are also discussed in the video. ``Reactive calculations''
appears to be a feature that defines how calculations can be performed
based on input values. ``Render plot'' explains how to create and
display graphics. This shows how you can create different types of plots
interactively using plot libraries such as Matplotlib. The video also
provides examples using different graphics packages, for example showing
how packages such as Seaborn, Plotnine, Pandas, Hollowvies, xarray and
geopandas can be used. It looks like this video might offer useful
information to those interested in data visualization using Shiny and
Python.

\begin{center}\rule{0.5\linewidth}{0.5pt}\end{center}

\bookmarksetup{startatroot}

\hypertarget{three-r-posts}{%
\chapter{\textbar{} Three R Posts}\label{three-r-posts}}

\hypertarget{mastering-data-visualization-in-r}{%
\subsection{1- Mastering Data Visualization in
R}\label{mastering-data-visualization-in-r}}

This R data visualization guide provides an overview of the various
techniques, libraries, and best practices for creating visually stunning
visuals that effectively communicate your data insights. You will be
better equipped to explore, analyze, and present your data findings in a
compelling and engaging manner if you master data visualization in R.
Whether you're a seasoned data analyst or just starting out, the world
of data visualization in R offers limitless opportunities to maximize
the value of your data.

\href{https://medium.com/@HalderNilimesh/mastering-data-visualization-in-r-a-comprehensive-guide-to-creating-stunning-visuals-and-f733564a8a41}{\emph{Post
Link}}

\hypertarget{human-resources-analytics-exploring-employee-data-in-r}{%
\subsection{2- Human Resources Analytics: Exploring Employee Data in
R}\label{human-resources-analytics-exploring-employee-data-in-r}}

The focus of this is on analyzing recruiting data, particularly in the
context of the sales department and the various recruiting channels from
which employees were hired. The R code includes loading the dataset,
providing an overview of the data, and identifying the recruiting
sources used by the company. Attrition rates, or the frequency of
employees leaving the company, are also examined by recruiting sources,
allowing for the identification of channels with both high and low
attrition rates. Data visualizations in the form of bar charts are used
to present the results, making it easier to compare and interpret the
findings.

Visualizing the sales performance differences The last step in the HR
analytics process is to test and plot the results. For now, you'll focus
on visualizing the data from the previous exercises. You'll be making a
bar chart so you can more easily see the average sales quota attainment
for each recruiting channel.

\bookmarksetup{startatroot}

\hypertarget{load-the-ggplot2-package}{%
\chapter{Load the ggplot2 package}\label{load-the-ggplot2-package}}

library(ggplot2)

\bookmarksetup{startatroot}

\hypertarget{plot-the-bar-chart}{%
\chapter{Plot the bar chart}\label{plot-the-bar-chart}}

ggplot(avg\_sales, aes(x = recruiting\_source, y =
avg\_sales\_quota\_pct)) + geom\_col()

\href{https://rpubs.com/alifrady/HR_analysis}{\emph{For More}}

\hypertarget{section}{%
\subsection{3-}\label{section}}

Th

\href{https://medium.com/@HalderNilimesh/mastering-data-visualization-in-r-a-comprehensive-guide-to-creating-stunning-visuals-and-f733564a8a41}{\emph{Post
Link}}



\end{document}
